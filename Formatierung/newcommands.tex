%  Neue Befehle:
%  * Version 1.0
%  * Samir Charif
%  * s.charif@tu-braunschweig.de
%  * Student, TU Braunschweig
%  * Mai 2022
%




%% ++++++++++++++++++++++++++++++++++++++++++++++++++++++++++++
%% Befehlre fürs hovern
%% ++++++++++++++++++++++++++++++++++++++++++++++++++++++++++++
% to use reference put the glsc.cwl in AppData\Roaming\texstudio\completion\user (if file is lost, then create one with equivalent name and write '/glsc{label}#r' into it), analog für die anderen Befehle mit \pdftooltip
%restart TexStudio
% after that go to 'Configurate TexStudio' (tab Options) and select the Completement Tab. Check the new file in that list and its done. 
\newcommand*{\glsh}[1]{\pdftooltip{\gls*{#1}}{\glsentrydesc{#1}}}
% "sbu" macro : shortcut for upright indices (ISO/NIST standard
% operatorfont for main document math font (serif or sans)
%\newcommand*{\sbuc}[1]{\pdftooltip{\sbu*{#1}}{_{\operatorfont{#1}}}}%
% ---------- glsvisub ----------------------------------------------------
% show $a_{b_c}$ with link to "a", variable "b" and subscript "sub.c"
% 4 arguments : 3 mandatory arguments and optional 1st and 2nd argument
% for adding accent on "a" and "b"
%\newcommand*{\glsvisubh}[#1]{\pdftooltip{\glsvisub*{#1}}{\glosmath@glsvisubRelay{#1}}}%s

% achtung funktioniert nur mit nohyperlinks im usepackage des acronym paketes, * zum unterdrücken hier leider nicht wirksam
\newcommand*{\ac}[1]{\pdftooltip{\acrshort*{#1}}{\acrlong{#1}}}
% da die Funktion von \ac, dass beim ersten Aufruf die lange UND Kurze version abgebildet wird durch die redefinition abgeschaltet ist
% bei erster Nutzung wird daher dieser Befehle benötigt
\newcommand*{\acfirst}[1]{\pdftooltip{\acrfull*{#1}}{\acrlong{#1}}}


% Bsp. für neue Mathe Befehle 
\newcommand{\R}{\mathbb{R}}
\newcommand{\N}{\mathbb{N}}
\newcommand{\diag}{\ensuremath{\text{diag}}}


%% Bildformatierung
\newlength\figureheight
\newlength\figurewidth

%% Textbefehle
%\newcommand{\markup}[1]{\textbf{#1}}	% Fettschreiben

\renewcommand*{\glscatnamefmt}[1]{\textbf{#1}} % category text format


\RequirePackage{iftex}
\ifluatex
\RequirePackage{luatex85}
\fi

