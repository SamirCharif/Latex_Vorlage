%% ++++++++++++++++++++++++++++++++++++++++++++++++++++++++++++
%% Dekblatt, Wurzel des Dokuments
%% ++++++++++++++++++++++++++++++++++++++++++++++++++++++++++++
%
%  Daten der Arbeit:
%  * Version 1.0
%  * B. Ing. Samir Charif, s.charif@tu-braunschweig.de
%  * Student, TU Braunschweig
%  * Mai 2022
%
%  Für Hauptseminare, Studienarbeiten, Diplomarbeiten
%
%  Autor           : Max Mustermann
%  Letzte Änderung : 04.12.2021
%
% Hier in die zweite geschweifte Klammer jeweils
% die persönlichen Daten und das Thema der Arbeit eintragen:

\newcommand{\artderausarbeitung}{Masterarbeit}
\newcommand{\namedesautors}{Samir Charif}
\newcommand{\themaderarbeit}{Beobachtung der Aushärtungsreaktion von
	Epoxidharz durch Anpassung von Modellparametern
	an gemessene elektrische Impedanzspektren}
\newcommand{\hochschule}{TU Braunschweig}
\newcommand{\studiengang}{Master Maschinenbau}
\newcommand{\martrikelnummer}{4992228}
\newcommand{\bearbeitungszeit}{6 Monate}
\newcommand{\jahrderabgabe}{2022}
\newcommand{\monatderabgabe}{02}
\newcommand{\tagderabgabe}{25}
\newcommand{\erstpruefer}{Prof. Dr.-Ing. Michael Sinapius (\hochschule - ima)}
\newcommand{\zweitpruefer}{M.Sc. Alexander Kyriazis (\hochschule - ima)}


%% ++++++++++++++++++++++++++++++++++++++++++++++++++++++++++++
%% PDF Metadaten definieren
%% ++++++++++++++++++++++++++++++++++++++++++++++++++++++++++++
\hypersetup{
	pdftitle={\themaderarbeit},
	pdfauthor={\namedesautors},
	pdfkeywords={\artderausarbeitung; \hochschule; Elektrodenpolarisation;
		Dipolrelaxation; Ionenleitfähigkeit;
	}, % auf Wunsch Schlagwörter anpassen
	pdfdate={\jahrderabgabe-\monatderabgabe-\tagderabgabe},
	pdftype={\artderausarbeitung},
	pdfsubject={\abstracteng},% auf wunsch ändern, für Bibliografie ist Englisch aber immer sinnvoller
}