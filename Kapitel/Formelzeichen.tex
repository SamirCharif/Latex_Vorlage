% Formelverzeichnis

\usepackage[
nonumberlist, %keine Seitenzahlen anzeigen
toc,          %Einträge im Inhaltsverzeichnis
section=chapter]      %im Inhaltsverzeichnis auf section-Ebene erscheinen
{glossaries}
\usepackage{hyperref}

\newglossary[slg]{symbolslist}{syi}{syg}{Symbolverzeichnis}
\renewcommand*{\glspostdescription}{}
\makeglossaries

%Eigener Gloassarystyle
\setglossarystyle{long3colheader}
\newglossarystyle{MyStyle}{
%  \glossarystyle{long3colheader} % Benötigt?
  \renewenvironment{theglossary}
  {\begin{longtable}{lp{2cm}p{\glsdescwidth}}}
    {\end{longtable}}
  \renewcommand*{\glossaryheader}{\textbf{Symbol} & \textbf{Einheit} &
    \textbf{Beschreibung}\\[3ex]\endhead}% ÄNDERUNG: Kopf auf jeder Seite wiederholen
  \renewcommand*{\glossaryentryfield}[5]{%
    \glsentryitem{##1}\glstarget{##1}{##2} & ##4 & ##3  \\[1ex]}%
}

	%%% Beispiel!!
	% Aufruf folgendermaßen :
	
	% \gls{symb:phiy} oder \glsc{symb:phiy}
	
	% Eintrag folgendermaßen:
	
	% \newglossaryentry{symb:phiy}{
		% name={\ensuremath{\varphi_{y}}},
		% symbol={Grad},
		% description={Winkel der Kugel in y-Richtung},
		% sort=symbolphiy, 
		% type=symbolslist}
	
	%%% Begin der Aufzählung
	\newglossaryentry{symb:R}{
		name={\ensuremath{R}},
		symbol={$\Omega$},
		description={Elektrischer Widerstand},
		sort=symbolR, 
		type=symbolslist}

	\newglossaryentry{symb:Re}{
		name={\ensuremath{Re}},
		symbol={$\Omega$},
		description={Elektrischer eeee Widerstand},
		sort=symbolR, 
		type=symbolslist}